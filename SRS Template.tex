% !TEX program = lualatex

\documentclass[12pt,a4paper]{report}
\usepackage{caption}
\usepackage{amsmath}
\usepackage{pifont}
\usepackage{array}
\usepackage{multirow}
\usepackage{enumitem}
\usepackage{listings}
\usepackage[table, dvipsnames]{xcolor}
\usepackage{longtable}
\usepackage{amsfonts}
\usepackage{amssymb}
\usepackage{fontspec}
\setmainfont{Times New Roman}
\setsansfont{Arial}
\usepackage{adjustbox}
\usepackage{graphicx}
\usepackage{fancyhdr}
\usepackage[explicit]{titlesec} 
\usepackage{tocloft}
\usepackage[top=0.8in,bottom=0.8in,left=0.4in,right=0.4in]{geometry}
\usepackage{tikz}
\usetikzlibrary{positioning}
\usepackage[final]{pdfpages}
\usepackage{hyperref}
\hypersetup{
    colorlinks=true,
    linkcolor=blue,
    filecolor=magenta,      
    urlcolor=red,
}

%pagestyle-------------------------------------------------------------------------
\pagestyle{fancy}
\fancyhf{}
\lhead{\small \textit{Software Requirements Specification for <Project>}}
\rhead{[Optional Header]}
\fancyfoot[R]{Page \thepage} 
\fancyfoot[L]{\hyperlink{toc}{Back to Table Of Contents}}
\renewcommand{\headrulewidth}{0.4pt}
\renewcommand{\footrulewidth}{0.4pt}
\setlength{\headheight}{20pt}
\assignpagestyle{\chapter}{fancy}

\fancypagestyle{firstpage}
{
\lhead{\small \textit{Software Requirements Specification for <Project>}}
\rhead{[Optional Header]}
\chead{}
\rfoot{}
\lfoot{}
\setlength{\headheight}{20pt}
}

\fancypagestyle{tocpage}
{
\lhead{\small \textit{Software Requirements Specification for <Project>}}
\rhead{[Optional Header]}
\chead{}
\rfoot{}
\lfoot{}
\cfoot{\textbf{\color{ForestGreen}{\large texts in blue or red are clickable}}}
\setlength{\headheight}{20pt}
}

\fancypagestyle{secondpage}
{
\lhead{\small \textit{Software Requirements Specification for <Project>}}
\rhead{Optional Header}
\chead{}
\rfoot{}
\lfoot{}
\setlength{\headheight}{20pt}
}

\addtocontents{toc}{\protect\thispagestyle{tocpage}}

\newcommand*\chapterlabel{}
\titleformat{\chapter}
  {}
  {\chapterlabel{ }}{0pt}
  % begin the main image, shifted 5cm down from top
  {\begin{tikzpicture}[remember picture,overlay]
    % add grey filled bar, 1cm high, linewidth wide.
    \node(start)[]{};
    \node(end)[right of=start, yshift=1cm,xshift=17.8cm]{};
    \draw [draw=black,fill=darkgray] (start) rectangle (end);
     % add the title, shifted to be centered in bar
    \node[align=center] at (8.9,.5) {\color{white}{\sffamily\upshape\bfseries\huge{\thechapter\quad #1}}};
   \end{tikzpicture}
  }
\titlespacing*{\chapter} {0pt}{-20pt}{10pt}

\titleformat{\section}{\Large\bfseries}{}{0pt}{\thesection \quad #1}
\titleformat{\subsection}{\large\bfseries}{}{0pt}{\thesubsection \quad #1}

\newcommand\temp[1]{
\begin{tikzpicture}[remember picture,overlay]
    % add grey filled bar, 1cm high, linewidth wide.
    \node(start)[]{};
    \node(end)[right of=start, yshift=1cm,xshift=17.8cm]{};
    \draw [draw=black,fill=darkgray] (start) rectangle (end);
     % add the title, shifted to be centered in bar
    \node[align=center] at (9.5,.5) {\color{white}{\sffamily\upshape\bfseries\huge{#1}}};
\end{tikzpicture}
}


\parindent 0ex
\vfuzz=3pt 
\hfuzz=5pt 
\renewcommand{\arraystretch}{2}
\setlength{\arrayrulewidth}{.5mm}
\newcolumntype{M}[1]{>{\centering\arraybackslash}m{#1}}
\newcolumntype{L}[1]{>{\raggedright\arraybackslash}m{#1}}

\setlength\cftbeforetoctitleskip{-10pt} 
\setlength\cftaftertoctitleskip{10pt} 
\setlength{\cftsecnumwidth}{2em} 
\cftsetindents{subsec}{3.5em}{2.75em}
\newlength{\drop}

\renewcommand\cftchapfont{\large\bfseries}
\renewcommand\cftsecfont{\large}

\renewcommand\cftchappagefont{\large\bfseries}
\renewcommand\cftsecpagefont{\large}

\newcommand{\cmark}{\ding{51}}
\newcommand{\xmark}{\ding{55}}
\tolerance=1
\emergencystretch=\maxdimen
\hyphenpenalty=10000
\hbadness=10000

\linespread{1.1}
%----------------------------------------------------------------------------------

\begin{document}

%title page
\begin{titlepage}
\thispagestyle{firstpage}
\vspace*{-1.5em}
\rule{\textwidth}{.1cm}
\vspace{-1em} \\
\fontsize{35}{38}\textbf{Software Requirements \\ Specification} 
\vspace{3em} \\
\fontsize{24}{29}\textbf{for} 
\vspace{2em} \\
\fontsize{30}{38}\textbf{<Project>}
\vspace{3em} \\
\fontsize{22}{29}\textbf{Version X} 
\vspace{3em} \\
\fontsize{22}{29}\textbf{<Organization>} 
\vspace{3em} \\
\fontsize{22}{29}\textbf{<Date Created>} 
\end{titlepage}

\thispagestyle{tocpage}
\addtocontents{toc}{\protect\hypertarget{toc}{}}
\tableofcontents
\newpage
\temp{Revisions}
\phantomsection\addcontentsline{toc}{chapter}{Revisions}
\begin{center}
\setlength\arrayrulewidth{1.5pt}
\begin{tabular}{|M{2cm}|M{4cm}|M{7.6cm}|M{3.3cm}|}
        \hline 
        \rowcolor{gray!25}
        \textbf{Version} & \textbf{Primary Author(s)} & \textbf{Description of Version} & \textbf{Date Completed}\\
        \hline 
        Draft Type and Number & Full Name & Information about the revision. This table does
        not need to be filled in whenever a document is
        touched, only when the version is being upgraded. & dd-mm-yyyy \\ 
        \hline 
\end{tabular}  
\end{center}

\chapter{Project Introduction}
\large
\section{Team Members}
\textit{<TO DO: List down the team members and their assigned actor/processes>}
\section{Problem Statement}

\section{Objectives}
\section{Project Plan}

\chapter{System Overview}
\section{Description}
\textit{<\textcolor{ForestGreen}{Summarize the major functions the product must perform or must let the user perform. Details will
be provided in Section 3, so only a high level summary is needed here. Organize the functions to
make them understandable to any reader of the SRS. A picture of the major groups of related
requirements and how they relate, such as a top level data flow diagram or object class diagram, will
be effective.} \\
TO DO: Describe the major processes to be performed by the system and the actors involved in
each process.>}

\section{Actors}
\textit{<\textcolor{ForestGreen}{Identify the various actors that will interact with this product.} \\
TO DO: List the actors and the use cases/functions that involve each the actor>}

\section{Assumptions and Dependencies}
\textit{\textcolor{ForestGreen}{<List any assumed factors (as opposed to known facts) that could affect the requirements stated in
the SRS. These could include third-party or commercial components that you plan to use, issues
around the development or operating environment, or constraints. The project could be affected if
these assumptions are incorrect, are not shared, or change. Also identify any dependencies the
project has on external factors, such as software components that you intend to reuse from another
project.} \\
TO DO: Provide a short list of some major assumptions that might significantly affect your design.
For example, you can assume that your client will have 1, 2 or at most 50 Automated Banking
Machines. Every number has a significant effect on the design of your system. >}

\section{Use Case Diagrams}
<TO DO: Place the use case diagram here.>
\chapter{Basic Requirements}
\section{Actor 1}
\subsection{Use Case 1}
<TO DO: Describe the use case.>
\subsection{Use Case 2}
<TO DO: Describe the use case.> \\
\hspace*{3cm} . \\
\hspace*{3cm} . \\
\hspace*{3cm} .
\subsection{Use Case N}
<TO DO: Describe the use case.>

\section{Actor 2} 
\subsection{Use Case 1}
<TO DO: Describe the use case.>
\subsection{Use Case 2}
<TO DO: Describe the use case.> \\
\hspace*{3cm} . \\
\hspace*{3cm} . \\
\hspace*{3cm} .
\subsection{Use Case N}
<TO DO: Describe the use case.>

\section{Actor N}
\hspace*{1.5cm} . \\
\hspace*{1.5cm} . \\
\hspace*{1.5cm} .

\chapter{Specific Requirements}
\section{Class Diagrams / ERD}
\textit{<TO DO: Describe the classes and place the class diagram.>}

\section{Sequence Diagrams}
\subsection{Use Case 1}
\textit{<TO DO: Describe the sequence and place the sequence diagram.>} \\
\hspace*{3cm} . \\
\hspace*{3cm} . \\
\hspace*{3cm} . 
\subsection{Use Case N}

\chapter{Other Requirements}
\textit{<This section is \underline{\textbf{Optional}}. \textcolor{ForestGreen}{Define any other requirements not covered elsewhere in the SRS. This
might include database requirements, internationalization requirements, legal requirements, reuse
objectives for the project, and so on. Add any new sections that are pertinent to the project}.>}
\end{document}